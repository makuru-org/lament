%%  docs/README.tex - Introduction to the documentation.
%%
%%     Copyright (C) 2024-2025  Kıvılcım Defne Öztürk
%%
%% This program is free software: you can redistribute it and/or modify
%% it under the terms of the GNU General Public License as published by
%% the Free Software Foundation, either version 3 of the License, or
%% (at your option) any later version.
%%
%% This program is distributed in the hope that it will be useful,
%% but WITHOUT ANY WARRANTY; without even the implied warranty of
%% MERCHANTABILITY or FITNESS FOR A PARTICULAR PURPOSE.  See the
%% GNU General Public License for more details.
%%
%% You should have received a copy of the GNU General Public License
%% along with this program. If not, see <https://www.gnu.org/licenses/>.


\documentclass{amsbook}
\usepackage[utf8]{inputenc}


\title{LAMENT: Lefty Application, Modification, Editing and Notification Tool}
\author{S. L. S. Sacramentum}
\date{08 April 2025}

\begin{document}
    \maketitle
    \pagenumbering{roman}
    \tableofcontents
    \pagenumbering{arabic}
    \section{Introduction}
    This is the first booklet in the documentation of Lefty Framework, which covers the documentation for LAMENT.
    LAMENT is the reference implementation of Sacramentum's Configuration Theory. Configuration theory describes
    how a computer system is configured and how this configuration can be applied to fleets of machines in a
    uniform way. It builds upon the ideas presented in Eelco Dolstra's thesis \textbf{\textit{The Purely Functional Software Deployment Model}} and
    makes them more accessible to the end users.

    \textit{Configuration Theory} presents us how computer systems are configured, how changes are applied, and more importantly,
    \textit{how declarative and immutable configuration can live together with plain old mutable configuration}.
    \newpage
    \section{License}
    \newpage
    \section{Contributing}
    \section{Contributions to LAMENT}\label{contributions-to-lament}

\subsection{Preamble}\label{preamble}

Free and open source software development is a collaborative job.
Various people work on the source code, and different people have
different styles of programming. While we value diversity a lot here,
the presence of a unified coding style is essential for everyone to
provide a more comfortable and convenient environment to collaborate.
Therefore, decisions about the style of the code, the way the pull
requests are described, and the way pull requests are merged should be
standardised.

\subsection{General Rules}\label{general-rules}

\begin{itemize}
\tightlist
\item
  Abide by the \href{CODE_OF_CONDUCT.md}{Code of Conduct}. We are very
  strict about this one. We will not tolerate harassment of any
  contributors here.
\end{itemize}

\subsection{Coding style}\label{coding-style}

\begin{itemize}
\tightlist
\item
  We generally follow the
  \href{https://github.com/luarocks/lua-style-guide}{style guidelines of
  LuaRocks}.
\item
  However we have one strict enforcement, use \texttt{local\ function}s
  whenever possible.
\end{itemize}

\subsection{Pull requests}\label{pull-requests}

\begin{itemize}
\tightlist
\item
  Use concise titles for pull requests and describe what it does well.
\item
  (The applicability of this article to forks on external servers not
  guaranteed) GitHub gives us a nice feature called \emph{tags} on pull
  requests. Use the appropriate tags for your pull requests.
\item
  And that's it you will have a large chance for it being merged.
\end{itemize}

\subsection{Commits (for project
administrators)}\label{commits-for-project-administrators}

\begin{itemize}
\tightlist
\item
  \textbf{VERY IMPORTANT:} The only commits on the \texttt{main} branch
  should be created by merges, unless some sort of apocalypse happens.
\item
  \textbf{Always create branches} for your commits and make a
  \textbf{pull request} to the \texttt{main} branch.
\end{itemize}

\subsection{Final notes}\label{final-notes}

\begin{itemize}
\tightlist
\item
  Do not be afraid to ask whenever you need help. We have both
  \href{https://github.com/Sparkles-Laurel/lament/discussions/10}{GitHub
  Discussions} and a
  \href{https://matrix.to/\#/\#lament-contrib:platypus-sandbox.com}{Matrix
  room}
\end{itemize}

    \newpage
    \section{Code of Conduct}
    \section{Contributor Covenant Code of
Conduct}\label{contributor-covenant-code-of-conduct}
\subsection{Our Pledge}\label{our-pledge}

We as members, contributors, and leaders pledge to make participation in
our community a harassment-free experience for everyone, regardless of
age, body size, visible or invisible disability, ethnicity, sex
characteristics, gender identity and expression, level of experience,
education, socio-economic status, nationality, personal appearance,
race, religion, or sexual identity and orientation.

We pledge to act and interact in ways that contribute to an open,
welcoming, diverse, inclusive, and healthy community.

\subsection{Our Standards}\label{our-standards}

Examples of behavior that contributes to a positive environment for our
community include:

\begin{itemize}
\item
  Demonstrating empathy and kindness toward other people
\item
  Being respectful of differing opinions, viewpoints, and experiences
\item
  Giving and gracefully accepting constructive feedback
\item
  Accepting responsibility and apologizing to those affected by our
  mistakes, and learning from the experience
\item
  Focusing on what is best not just for us as individuals, but for the
  overall community
\end{itemize}

Examples of unacceptable behavior include:

\begin{itemize}
\item
  The use of sexualized language or imagery, and sexual attention or
  advances of any kind
\item
  Trolling, insulting or derogatory comments, and personal or political
  attacks
\item
  Public or private harassment
\item
  Publishing others' private information, such as a physical or email
  address, without their explicit permission
\item
  Other conduct which could reasonably be considered inappropriate in a
  professional setting
\end{itemize}

\subsection{Enforcement
Responsibilities}\label{enforcement-responsibilities}

Community leaders are responsible for clarifying and enforcing our
standards of acceptable behavior and will take appropriate and fair
corrective action in response to any behavior that they deem
inappropriate, threatening, offensive, or harmful.

Community leaders have the right and responsibility to remove, edit, or
reject comments, commits, code, wiki edits, issues, and other
contributions that are not aligned to this Code of Conduct, and will
communicate reasons for moderation decisions when appropriate.

\subsection{Scope}\label{scope}

This Code of Conduct applies within all community spaces, and also
applies when an individual is officially representing the community in
public spaces. Examples of representing our community include using an
official e-mail address, posting via an official social media account,
or acting as an appointed representative at an online or offline event.

\subsection{Enforcement}\label{enforcement}

Instances of abusive, harassing, or otherwise unacceptable
behavior may be reported to the community leaders responsible for
enforcement at lament.abuse@yigitovski.com. All complaints will be
reviewed and investigated promptly and fairly.

All community leaders are obligated to respect the privacy and security
of the reporter of any incident.

\subsection{Enforcement Guidelines}\label{enforcement-guidelines}

Community leaders will follow these Community Impact Guidelines in
determining the consequences for any action they deem in violation of
this Code of Conduct:

\subsubsection{1. Correction}\label{correction}

\textbf{Community Impact}: Use of inappropriate language or other
behavior deemed unprofessional or unwelcome in the community.

\textbf{Consequence}: A private, written warning from community leaders,
providing clarity around the nature of the violation and an explanation
of why the behavior was inappropriate. A public apology may be
requested.

\subsubsection{2. Warning}\label{warning}

\textbf{Community Impact}: A violation through a single incident or
series of actions.

\textbf{Consequence}: A warning with consequences for continued
behavior. No interaction with the people involved, including unsolicited
interaction with those enforcing the Code of Conduct, for a specified
period of time. This includes avoiding interactions in community spaces
as well as external channels like social media. Violating these terms
may lead to a temporary or permanent ban.

\subsubsection{3. Temporary Ban}\label{temporary-ban}

\textbf{Community Impact}: A serious violation of community standards,
including sustained inappropriate behavior.

\textbf{Consequence}: A temporary ban from any sort of interaction or
public communication with the community for a specified period of time.
No public or private interaction with the people involved, including
unsolicited interaction with those enforcing the Code of Conduct, is
allowed during this period. Violating these terms may lead to a
permanent ban.

\subsubsection{4. Permanent Ban}\label{permanent-ban}

\textbf{Community Impact}: Demonstrating a pattern of violation of
community standards, including sustained inappropriate behavior,
harassment of an individual, or aggression toward or disparagement of
classes of individuals.

\textbf{Consequence}: A permanent ban from any sort of public
interaction within the community.

\subsection{Attribution}\label{attribution}

This Code of Conduct is adapted from the \href{https://www.contributor-covenant.org}{Contributor Covenant},
version 2.0, available at \href{https://www.contributor-covenant.org/version/2/0/code\_of\_conduct.html}{version/2/0/code\_of\_conduct.html}.

Community Impact Guidelines were inspired by
\href{https://github.com/mozilla/diversity}{Mozilla's code of conduct
enforcement ladder}.

For answers to common questions about this code of conduct, see the FAQ
at \href{\url{https://www.contributor-covenant.org/faq}}{https://www.contributor-covenant.org/faq}.
Translations are available
at \href{https://www.contributor-covenant.org/translations}{https://www.contributor-covenant.org/translations}.

\end{document}
