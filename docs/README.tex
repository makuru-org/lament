%%     Copyright (C) 2024-2025  Kıvılcım Defne Öztürk
%%
%% This program is free software: you can redistribute it and/or modify
%% it under the terms of the GNU General Public License as published by
%% the Free Software Foundation, either version 3 of the License, or
%% (at your option) any later version.
%%
%% This program is distributed in the hope that it will be useful,
%% but WITHOUT ANY WARRANTY; without even the implied warranty of
%% MERCHANTABILITY or FITNESS FOR A PARTICULAR PURPOSE.  See the
%% GNU General Public License for more details.
%%
%% You should have received a copy of the GNU General Public License
%% along with this program.  If not, see <https://www.gnu.org/licenses/>.


\documentclass{amsbook}
\usepackage[utf8]{inputenc}


\title{LAMENT: Lefty Application, Modification, Editing and Notification Tool}
\author{S. L. S. Sacramentum}
\date{08 April 2025}

\begin{document}
    \maketitle
    \pagenumbering{roman}
    \tableofcontents
    \pagenumbering{arabic}
    \section{Introduction}
    This is the first booklet in the documentation of Lefty Framework, which covers the documentation for LAMENT.
    LAMENT is the reference implementation of Sacramentum's Configuration Theory. Configuration theory describes
    how a computer system is configured and how this configuration can be applied to fleets of machines in a
    uniform way. It builds upon the ideas presented in Eelco Dolstra's thesis \textbf{\textit{The Purely Functional Software Deployment Model}} and
    makes them more accessible to the end users.
\end{document}
